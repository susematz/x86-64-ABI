
\editornote{This chapter is used to document some features special to
  the \xARCH ABI.  The different sections might be moved to another
  place or removed completely.}


\section{GOT pointer and IP relative addressing}

\index{global offset table}
A basic difference between the \intelabi and the \xARCH ABI is the
way the GOT table is found.  The \intelabi, like (most) other processor
specific ABIs, uses a dedicated register (\reg{ebx}) to address the
base of the GOT table.  The function prologue of every function needs
to set up this register to the correct value.  The \xARCH processor
family introduces a new IP-relative addressing mode which is used in
this ABI instead of using a dedicated register.

On \xARCH the GOT table contains 64-bit entries.


\section{C++\label{section-cpp}}

For the \textindex{C++} ABI we will use the IA-64 C++ ABI and instantiate it
appropriately.  The current draft of that ABI is available at:\\
\url{http://www.codesourcery.com/cxx-abi/}

%%% We would like to include the fortran text here, but we can't nest
%%% includes.  Hence, do it in the main file.
%%% \include{fortran}

%%% Local Variables:
%%% mode: latex
%%% TeX-master: "abi"
%%% End:
