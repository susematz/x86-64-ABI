\chapter{x86-64 Linux Kernel Conventions}


This chapter is informative only.

\section{Calling Conventions}

The Linux \xARCH kernel uses internally the same calling conventions as user-level
applications (see section \ref{sec-calling-conventions} for details).
User-level applications that like to call system calls should use the
functions from the C library.  The interface between the C library and
the Linux kernel is the same as for the user-level applications with
the following differences:
\begin{enumerate}
\item User-level applications use as integer registers for passing the
  sequence \RDI, \RSI, \RDX, \RCX, \reg{r8} and \reg{r9}.  The kernel
  interface uses \RDI, \RSI, \RDX, \reg{r10}, \reg{r8} and \reg{r9}.
\item A system-call is done via the \code{syscall} instruction.  The
  kernel destroys registers \RCX and \reg{r11}.
\item The number of the syscall has to be passed in register \RAX.
\item System-calls are limited to six arguments, no argument is passed
  directly on the stack.
\item Returning from the \code{syscall}, register \RAX contains the
  result of the system-call.  A value in the range between -4095 and
  -1 indicates an error, it is \code{-errno}.
\item Only values of class INTEGER or class MEMORY are passed to the
  kernel.
\end{enumerate}





%%% Local Variables:
%%% mode: latex
%%% TeX-master: "abi"
%%% End:
